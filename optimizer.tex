\section{Trajectory Optimizer}
\label{sec:optimizer}
Prior work has also proposed to plan through learned models via differentiation, though not with visual inputs~\cite{deep_mpc}. We instead use a stochastic, sampling-based planning method~\cite{cem-rk-13,foresight}, which we extend to handle a mixture of continuous and discrete actions

The role of the optimizer is to find actions sequences $a_{t:t+T}$ which minimize the sum of the costs $c_{t:t+T}$.

In the visual MPC setting the action sequences found by the optimizer can be very different between execution times steps $\tau$ (not to be confused with prediction time steps $t$). For example at one time step $\tau$ the optimizer might find a pushing action leading towards the goal and in the next time step $\tau+1$ it determines a grasping action to be optimal to reach the goal. Naive replanning at every time step can then result in alternating between a pushing and a grasping attempt indefinitely causing the agent to get stuck and not making any progress towards to goal. 

We show that we can resolve this problem by modifying the sampling distribution of the first iteration of CEM so that the optimizer commits to the plan found in the previous time step. In prior work \cite{sna} the sampling distribution at first iteration of CEM is chosen to be a Gaussian with diagonal covariance matrix and zero mean. We instead use the best action sequence found in the optimization of the previous time step as the mean. Since this action sequence is optimized for the previous time step we only use the values $a_{1:T}$ and omit the first action, where $T$ is the prediction horizon. To sample actions close to the action sequence from the previous time step we reduce the entries of the diagonal covariance matrix for the first $T-1$ time steps. It is crucial that the last entry of the covariance matrix at the end of the horizon is not reduced otherwise no exploration could happen for the last time step causing poor performance at later time steps.