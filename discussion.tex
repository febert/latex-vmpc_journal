\section{Discussion}
A natural question concerning the presented method is whether predicting in image space or a learned latent space is better for the purpose of generalizable robotic control.
%%SL.10.16: But we don't study this question...
While a latent space has the advantage of likely being a more compact representation, potentially being easier to predict while mostly encoding information relevant to the prediction task, it is an open problem \emph{how} to automatically extract such a latent space. Furthermore the capability to generate predictions in image space helps immensely with interpreting and debugging the system. Of course decoding a latent space prediction back to images for the purpose of interpretation would likely be feasible as well. 
%%SL.10.16: This doesn't belong right after experiments, it's not stating a conclusion that is based on any evidence that is actually in the paper.

Another potential advantage of latent space representations could be to handle \emph{partially observed} settings, for example tasks where an object is occluded for extended periods of time. However the recurrent video-prediction model is also able to handle some cases of partial observability, through the use of temporal skip-connections and by maintaining a hidden state in the convolutinal LSTM.

%%SL.10.16: I would suggest just cutting this section. It's completely out of place and has nothing to do with what's in the paper. You could perhaps condense it into 2-3 sentences in the Discussion and Future Work Section

%%SL.10.16: But more importantly, you *really* need to discuss the conclusions to be drawn from the experiments!
